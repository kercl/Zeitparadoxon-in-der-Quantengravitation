\documentclass{scrartcl}
\usepackage[utf8]{inputenc}
\usepackage[T1]{fontenc}
\usepackage[ngerman,german]{babel}
\usepackage{amsmath}
\usepackage{amsthm}
\usepackage{amssymb}
\usepackage{sectsty}
\usepackage{tikz}
\usepackage{bbold}
\usepackage{enumerate}
\usepackage{ifthen}
\usepackage{breqn}
\usetikzlibrary{shapes,shadows,arrows,decorations.markings}
\usetikzlibrary{decorations.pathreplacing,calc}


\newtheorem{theorem}{Satz}
\newtheorem{lemma}{Lemma}
\newtheorem{definition}{Definition}
\newtheorem{proposition}{Proposition}

\newcommand{\inftyint}{\int_{-\infty}^{\infty}}
\newcommand{\intR}{\int_{\mathbb{R}}}
\newcommand{\intRRR}{\int_{\mathbb{R}^3}}
\newcommand{\opd}[2][\vphantom{1}]{\!\operatorname{d}^{#1}\! #2\ }
\newcommand{\opdpath}{\opd{\mu(p)}}
\newcommand{\real}{\operatorname{Re}}
\newcommand{\imag}{\operatorname{Im}}
\newcommand{\residue}{\operatorname{Res}}
\newcommand{\veps}{\varepsilon}
\newcommand{\vx}{\vec{x}}
\newcommand{\vp}{\vec{p}}
\newcommand{\vacuumt}{\langle 0 |}
\newcommand{\vacuum}{| 0 \rangle}
\newcommand{\Res}{\operatorname{Res}}
\newcommand{\inHS}{{}^{(3)}\!}
\newcommand{\VSthree}{\mu(\mathcal{S}^3)}

\title{Das Problem der Zeit in der Quantengravitation}
\subtitle{Im Rahmen des Seminars ,,Ausgewählte Probleme der Quantentheorie``}
\author{Clemens Kerschbaumer}
\date{\today}

\begin{document}
	\maketitle
	\section{Überblick über mathematische Konzepte der Differentialgeometrie}
		Eine Mannigfaltigkeit kann intuitiv definiert werden, als ein Objekt, das sich lokal, so verhällt wie ein $\mathbb{R}^n$. Zum Beispiel
		ist in ihrer Gesamtheit die Erde (mehr oder weniger) eine Kugel. Jedoch sieht sie von unserer Perspektive aus betrachtet verhältnismäßig flach aus.
		
		Die Erde kann als eingebettete Fläche in einem dreidimensionalen Raum angesehen werden.
		Allgemein bezeichnet man eine eingebettete Manigfaltigkeit mit Dimension $m$ Hyperfläche
		(im folgenden bezeichnet $\Sigma$ immer eine Hyperfläche), falls die die Dimension $m=n-1$. 
		
		\begin{paragraph}{Die Metrik:} Metriken sind im allgemeinen Abbildungen, die Distanzen messen. In der allgemeinen
			Relativitätstheorie wird dieses Konzept bekanntlich verallgemeinert, da die Metrik hier auch negative
			Werte annehmen kann. Der Abstand zwischen zwei Punkten auf einer Mannigfaltigkeit kann als die kürzeste Länge 
			aller möglichen Wege (in Analogie zur vertrauten euklidischen Metrik) angegeben werden, sogenannte Geodäten.
			
			Formal lässt sich das mithilfe des Metriktensors beschreiben (man hört oft, dass beim Metriktensor auch nur von
			der Metrik gesprochen wird). Der Metriktensor kann sich intuitiv als eine Vorschrift vorgestellt werden, die an jedem
			Punkt der Mannigfaltigkeit angiebt wie sehr die uns vertrauten Längen gestaucht bzw gestreckt werden, wenn 
			man entlang eines Weges geht. Infinitesimal schreibt man oft
			\begin{align}
				{\rm d}s^2=g_{\mu\nu} {\rm d}x^\mu{\rm d}x^\nu.
				\label{equ:infline}
			\end{align}
			Das legt nahe, dass man, nach Wahl einer geeigneten Basis, den Metriktensor als Matrix schreiben kann.
			
			Als Beispiel kann man sich Zylinderkoordinaten vorstellen. Der Zylinder mit fixem Radius $r$ ist eine 
			zweidimensionale Mannigfaltigkeit mit der Parametrisierung:
			\begin{align}
				\vec{x}(\varphi,z)=\left( 
				\begin{array}{c}
				r\cos(\varphi)\\
				r\sin(\varphi)\\
				z
				\end{array}
				\right)
			\end{align}
			\begin{figure}
				\centering
				


\begin{tikzpicture}[y=0.80pt, x=0.8pt,yscale=-1, inner sep=0pt, outer sep=0pt]
\begin{scope}[shift={(-69.44543,-862.60218)}]
  \path[cm={{0.86136,-0.508,0.508,0.86136,(-435.02941,302.9781)}},draw=black,line
    join=miter,line cap=butt,miter limit=4.00,line width=0.400pt]
    (410.0000,895.4872) .. controls (410.0000,905.4973) and (377.5406,913.6122) ..
    (337.5000,913.6122) .. controls (297.4594,913.6122) and (265.0000,905.4973) ..
    (265.0000,895.4872) .. controls (265.0000,885.4770) and (297.4594,877.3622) ..
    (337.5000,877.3622) .. controls (377.5406,877.3622) and (410.0000,885.4770) ..
    (410.0000,895.4872) -- cycle;
  \path[draw=black,line join=miter,line cap=butt,miter limit=4.00,line
    width=0.400pt] (249.0882,941.3095) -- (302.4279,1031.7522);
  \path[draw=black,line join=miter,line cap=butt,miter limit=4.00,line
    width=0.400pt] (373.9569,867.3734) -- (427.2966,957.8160);
  \path[draw=black,line join=miter,line cap=butt,miter limit=4.00,line
    width=0.400pt] (426.8804,957.3389) .. controls (431.9655,965.9612) and
    (408.1287,989.4403) .. (373.6393,1009.7808) .. controls (339.1500,1030.1213)
    and (307.0685,1039.6208) .. (301.9834,1030.9985);
  \path[draw=black,dash pattern=on 3.20pt off 1.60pt,line join=miter,line
    cap=butt,miter limit=4.00,line width=0.400pt] (426.9439,957.4466) .. controls
    (421.8588,948.8242) and (389.7773,958.3237) .. (355.2880,978.6642) .. controls
    (320.7986,999.0048) and (296.9618,1022.4838) .. (302.0469,1031.1062);
  \path[draw=black,miter limit=4.00,line width=0.400pt,rounded corners=0.0000cm]
    (69.6954,958.1198) rectangle (172.3740,1044.7269);
  \path[draw=black,fill=black,miter limit=4.00,line width=0.800pt]
    (85.9321,965.9745) -- (82.8610,967.9948) -- (85.9321,960.9237) --
    (89.0032,967.9948) -- cycle;
  \path[draw=black,line join=miter,line cap=butt,miter limit=4.00,line
    width=0.400pt] (85.9892,966.6255) -- (85.9892,1040.9980);
  \path[draw=black,line join=miter,line cap=butt,miter limit=4.00,line
    width=0.400pt] (74.1199,1029.7600) -- (156.9524,1029.7600);
  \path[draw=black,fill=black,miter limit=4.00,line width=0.800pt]
    (157.3312,1029.7600) -- (155.3109,1026.6889) -- (162.3820,1029.7600) --
    (155.3109,1032.8310) -- cycle;
  \path[draw=black,line join=miter,line cap=butt,line width=0.800pt]
    (126.2691,952.3571) .. controls (165.9766,870.8115) and (191.3674,1010.4302)
    .. (262.6397,988.7226);
  \path[draw=black,fill=black,miter limit=4.00,line width=0.800pt]
    (263.0056,988.4181) -- (260.0930,986.1753) -- (267.7794,986.7684) --
    (262.0992,991.9806) -- cycle;
  \path[fill=black] (93.186569,974.58044) node[above right] (text3863) {$z$};
  \path[fill=black] (148.99751,1019.7849) node[above right] (text3867) {$\phi$};
  \path[fill=black] (165.66502,928.61853) node[above right] (text3871)
    {$\vec{x}(\phi,z)$};
\end{scope}

\end{tikzpicture}

				\caption{Zylinder als Mannigfaltigkeit.}
			\end{figure}
			Wenn man in Zylinderkoordinaten wechselt, ist ja bekannt, dass das infinitesimale Linienelement nichts anderes als
			\begin{align*}
				{\rm d}s^2=r{\rm d}\varphi^2 + {\rm d}z^2
			\end{align*}
			ist. Daraus lässt sich, vergleicht man diese Gleichung mit (\ref{equ:infline}), die zugehörige Matrix in der Standardbasis ablesen:
			\begin{align*}
				g=\left(\begin{array}{ccc} r^2 & 0\\0 & 1 \end{array}\right).
			\end{align*}
			Man sieht hier deutlich: Wenn man einen konstanten Winkel abfährt, wird der zurückgelegte Weg größer falls $r>1$ (der Metriktensor streckt also
			Wege entlang einer Änderung in $\varphi$ entsprechend dem Radius) bzw kleiner falls $r<1$ verglichen mit dem Weg im kartesischen Koordinatensystem,
			das sich aus Winkel und Höhenangabe zusammensetzt.
		\end{paragraph}
		\begin{paragraph}{Die kovariante Ableitung:}
			Man nehme als Beispiel wieder eine Fläche, eingebettet im $\mathbb{R}^3$. Entlang dieser Fläche soll eine Kurve verlaufen. Wie üblich 
			kann man die Ableitung der Kurve an einem beliebigen Punkt bilden und erhällt den Tangentenvektor. Alle möglichen Tangentenvektoren,
			aller möglichen Kurven konstituieren an jedem Punkt der ebene zusammen eine Menge, den sogenannten Tangentialraum $T_p(M)$
			(oder in diesem Fall, Tangentialebene). Tatsächlich kann man zeigen, dass es sich hierbei um einen Vektorraum handelt. Die Basisvektoren
			sind einfach nur die Ableitungen der Parametrisierungen. Wieder dienen Zylinderkoordinaten als anschauliches Beispiel. 
			Die Ableitung der Parametrisierung ergeben
			\begin{align*}
				\vec{e}_\varphi&=\left(\begin{array}{c} -r\sin(\varphi)\\ r\cos(\varphi)\\ 0\end{array}\right)\\
				\vec{e}_z&=\left(\begin{array}{c} 0\\0\\1\end{array}\right)
			\end{align*}
			Nun kann man bekanntlich auch die Ableitung von Vektoren bilden. Geht man naiv vor und nimmt wie zuvor die reguläre Ableitung 
			führt das auf gekrümmten Flächen dazu, dass der Vektor zum einen in der Tangentialebene verändert (für ein kleines ${\rm d}x\in T_p(M)$)
			und zugleich normal zur Ebene (${\rm d}n\not \in T_p(M)$). Die Normalveränderung ist jedoch nicht ,,physikalisch`` (bzw. sinnvoll miteinzubeziehen)
			da der Punkt gar nicht anders kann, als der Ebene zu folgen. Er muss sich sozusagen nicht selbst mitdrehen, das erledigt die Zwangsbedingungen,
			die verlangt, dass die Kurve in der Fläche bleibt.
			
			Die Vektoren können mithilfe der Basisvektoren in der Form
			\begin{align}
				v=a\vec{x}_u+b\vec{x}_v
			\end{align}
			geschrieben werden. Achtung: Das ist \textit{kein} dreidimensionaler Vektorraum, obwohl die Vektoren $\vec{x}_u$, in der Standardbasis ausgedrückt 3 Komponenten
			besitzen. Vielmehr kann man sich das Konstrukt als Unterraaum (genauer: affinen Unterraum) vorstellen. Konkret für die Kurve gilt, dass $a=\dot{u}$ und
			$b=\dot{v}$ (Kettenregel beim bilden des Tangentenvektors).
			
			Beschränkt man sich nur auf diesen Raum, kann man
			die Vektoren auch schreiben als
			\begin{align}
				v=\left(\begin{array}{c} a \\ b\end{array}\right).
			\end{align}
			Würde man nur die Komponenten von $v$ differenzieren lässt sich die Krümmung nicht miteinbeziehen. Daher muss man kurz wieder den Umgebungsraum
			betrachtet.
			Der im $\mathbb{R}^3$ liegenden Vektor ergibt differenziert
			\begin{align}
				\frac{{\rm d}}{{\rm d}t}(a\vec{x}_u+b\vec{x}_v)=&\dot{a}\vec{x}_u+\dot{b}\vec{x}_v\notag\\
				&+a\vec{x}_{uu}\dot{u}+a\vec{x}_{uv}\dot{v}+b\vec{x}_{vu}\dot{u}+b\vec{x}_{vv}\dot{v}.
				\label{equ:covderiv1}
			\end{align}
			Hier wurde die Kettenregel verwendet.
			Man will, wie oben besprochen nur den Teil, der sich -- mehr oder weniger -- im Raum abspielt (intrinsische Größe) isolieren. Dadurch, dass die 
			$\vec{x}_{uu}$, $\vec{x}_{uv}$ und $\vec{x}_{vv}$ Vektoren im $\mathbb{R}^3$ sind, lassen sie sich beispielsweise in der Basis $\vec{x}_{u},\ \vec{x}_{v}$
			und $n$ ausdrücken:
			\begin{align}
				\vec{x}_{uu}&=\Gamma^1_{11} \vx_u+\Gamma^2_{11}\vx_v+L_1 n,\\
				\vx_{uv}&=\Gamma^1_{12}\vx_u+\Gamma^2_{12}\vx_v +L_2 n,\\
				\vx_{vv}&=\Gamma^1_{22}\vx_u+\Gamma^2_{22}\vx_v+L_3 n.
			\end{align}
			Die $\Gamma^i_{jk}$ und $L_i$ sind a priori nur passend gewählte Faktoren.
			Setzt man dies in (\ref{equ:covderiv1}) ein und fasst zusammen erhällt man, wenn man die Normalkomponenten wie gewünscht ignoriert 
			\begin{align}
				\frac{{\rm D} v(t)}{{\rm d}t}=\nabla_{v(t)} v(t) :=& (\dot{a} + \Gamma^1_{11}a^2+\Gamma^1_{12}ab+\Gamma^1_{22}b^2)\vx_u+\notag\\
				+&(\dot{b} + \Gamma^2_{11}a^2+\Gamma^2_{12}ab+\Gamma^2_{22}b^2)\vx_v.
			\end{align}
			Diese Gleichung beschreibt wie sich ein Tangentialvektor entlang einer Kurve in einer gekrümmten Fläche im Tangentialraum verhällt, wenn man entlang dieser Kurve wandert.
			Das definiert die kovariante Ableitung. Wenn man entlang der Koordinatenachsen ableitet, kann man die kovariante Ableitung elegant mit der Indexnotation
			schreiben:
			\begin{align}
				\nabla_i v^j=\partial_i v^j + \Gamma^j_{ki}v^k,
			\end{align}
			wobei $i=1,2$. Diese Gleichungen (für jede Komponente) werden bald in einer etwas allgemeineren Form abgeleitet werden und tragen häufig den Namen ,,Gauss-Weingarten Gleichungen``.
		\end{paragraph}
		Für vierdimensionale Mannigfaltigkeiten lässt sich diese Definition verallgemeinern:
		\begin{align*}
			\nabla_\mu v^\nu=\partial_\mu v^\nu +\Gamma^\nu_{\rho\mu}v^\rho.
		\end{align*}
		Für die in der allgemeinen Relativitätstheorie üblichen kovarianten Ableitung (siehe Levi-Civita Zusammenhang) lassen sich die Christoffelsymbole
		über die Formel
		\begin{align}
			\Gamma^\mu_{\nu\rho}=\frac{1}{2}g^{\mu\sigma}(\partial_\rho g_{\sigma \nu}+\partial_\nu g_{\sigma \rho}-\partial_\sigma g_{\nu\rho}).
			\label{def:christ}
		\end{align}
		
		Die folgenden Abschnitte werden zur Ableitung der allgemeinen Form der Wheeler-DeWitt Gleichung benötigt. Für den im Vortrag
		besprochenen Minisuperspace sind diese Überlegungen nicht unbedingt notwendig. Es wird lediglich die Definition der extrinsischen Krümmung $K$ benötigt.
		\begin{paragraph}{Extrinsische Krümmung:} Neben der intrinsischen Krümmung existiert noch ein weiterer Begriff, die extrinsische Krümmung. Man betrachte
		dazu zuerst die auf Hyperebenen definierbare Abbildung (Weingartenabbildung):
		\begin{align*}
			L:\ T_p(\Sigma)&\rightarrow T_p(\Sigma) \\
				v&\mapsto \nabla_v n
		\end{align*}
		definieren, welche bis auf Vorzeichenwahl des Normalvektors eindeutig ist. $\nabla_v$ bezeichnet hierbei die kovariante Ableitung 
		des Umgebungsraumes.
		\begin{proposition}{Proposition:} $L$ ist sebstadjungiert (d.h. für ein inneres Produkt definiert im Umgebungsraum $\mathbb{R}^n$, gilt 
			$\forall u,v\in \mathbb{R}^n:\langle u,L(v)\rangle=\langle L(u),v\rangle)$ 
		\end{proposition}
		Die Definition der Weingartenabbildung und die damit verbundene Eigenschaft legen die Definition einer neuen Größe nahe.
		Intuitiv lässt sich die Krümmung einer Fläche messen, indem man die Änderungsrate der Tangentialebene (oder equivalent, des Normalenvectors misst).
		Mathematisch lässt sich das mittels der Weingartenabbildung ausdrücken:
		\begin{definition} Extrinisische Krümmung K:
			\begin{align}
				K:\ T_p(\Sigma)\times T_p(\Sigma)&\rightarrow \mathbb{R} \label{def:extrcurv} \\
					(v,u)&\mapsto -\langle v,L(u)\rangle.\notag
			\end{align}
		\end{definition}
		Im folgenden wird oft die einsteinsche Summenkonvention angewandt, was die Wahl einer Basis notwendig macht. In der Standardbasis
		ergibt sich daher für die Krümmung $K_{ij}=K(e_i,e_j)$. Desweiteren wird die Notation $\nabla_i:=\nabla_{e_i}$ genützt werden.

		Nun lässt sich die kovariante Ableitung, definiert auf der Hyperebene, mit der kovarianten Ableitung des Umgebungsraumes in Beziehung bringen.
		Offensichtlich lässt sich der Tangentialraum $T_p(\mathbb{R}^n)$ des Einbettungsraumes, schreiben als direkte Summe:
		\begin{align}
			T_p(\mathbb{R}^n) = \operatorname{span}(n)\oplus T_p(\Sigma).
		\end{align}
		Die kovariante Ableitung auf der Hyperebene ist definiert als die Projektion der kovarianten Ableitung des Umgebungsraumes auf die Tangentialräume
		des Hyperraumes $\inHS\nabla:=P\circ \nabla$, wobei
		\begin{align*}
			P:\ T_p(\mathbb{R}^n)&\rightarrow T_p(\Sigma)\\
				v&\mapsto v-|\langle v,n\rangle| n.
		\end{align*}
		
		\begin{theorem} Gauss-Weingarten Gleichungen:
			\begin{align}
				\nabla_i e_j=-K_{ij}n+{}^{(3)}\Gamma^k_{ij}e_k.
			\end{align}
		\end{theorem}
		\begin{proof}
			Aus der Leibnizregel ($\nabla_i \langle v,n\rangle =\langle v,\nabla_i n\rangle + \langle \nabla_i v,n\rangle$) und der Orthogonalität von $v$ und $n$ folgt
			\begin{align*}
				\langle v,\nabla_i n\rangle = -\langle \nabla_i v,n\rangle
			\end{align*}
			was man zusammen mit Definition der Krümmung $K_{ij}$ die Behauptung bestätigt:
			\begin{align*}
				(\inHS\nabla_i e_j)^\mu&=P^\mu_\nu\nabla_i\delta^\nu_j=(\delta^\mu_\nu + n^\mu n_\nu)\nabla_i \delta^\mu_j=\\
				&=(\nabla_i e_j)^\mu + n^\mu n_\nu\nabla_i e_j^\nu=(\nabla_i e_j)^\mu - n^\mu e_j^\nu\nabla_i n_\nu=(\nabla_i e_j - n K_{ij})^\mu.
			\end{align*}
		\end{proof}
		Damit lässt sich folgender Satz beweisen:
		\begin{theorem} Gauss-Codazzi Gleichungen:
			\begin{align}
				{R^0}_{ijk}&=\inHS\nabla_j K_{ik}-\inHS\nabla_k K_{ji} \\
				R_{mijk}&=-(K_{ij}K_{mk}-K_{ik}K_{mj})+\inHS R_{mijk}
			\end{align}
			\label{thm:gce}
		\end{theorem}
		\begin{proof}
			Mithilfe des vorherigen Lemmas ergibt sich:
			\begin{align*}
				\nabla_i\nabla_j e_k &= \nabla_i(K_{jk}n+\inHS\Gamma^m_{jk}e_m)=\\
									 &= K_{jk,i}n+K_{jk}\nabla_i n+\inHS\Gamma^m_{jk,i}e_m+\inHS\Gamma^m_{jk}\nabla_i e_m=\\
									 &= K_{jk,i}n+K_{jk}{K_i}^m e_m+\inHS\Gamma^m_{jk,i}e_m+\inHS\Gamma^m_{jk}(K_{im}n+\inHS\Gamma^l_{im}e_l)=\\
									 &= (K_{jk,i}+\inHS\Gamma^m_{jk}K_{im})n+(K_{jk}{K_i}^l+\inHS\Gamma^l_{jk,i}+\inHS\Gamma^m_{jk}\inHS\Gamma^l_{im})e_l.
			\end{align*}
			Eingesetzt in die Definition des Riemanntensors ergibt sich:
			\begin{align*}
				{R^\alpha}_{ijk}e_\alpha =&\nabla_i\nabla_j e_k-\nabla_j\nabla_i e_k=\\
										 =&(K_{jk,i}+\inHS\Gamma^m_{jk} K_{im}-K_{ik,j}-\inHS\Gamma^m_{ik}K_{jm})n+\\
										 &+(K_{jk}{K_i}^m-K_{ik}{K_j}^m+\inHS {R^m}_{ijk})e_m,
			\end{align*}
			wobei benutzt wurde, dass
			\begin{align*} 
				\inHS {R^m}_{ijk}=\inHS\Gamma^m_{jk,i}-\inHS\Gamma^m_{ik,j}+\inHS\Gamma^l_{jk}\inHS\Gamma^m_{il}-\inHS\Gamma^l_{ik}\inHS\Gamma^m_{jl}.
			\end{align*}
			Aus der Orthogonalität der $e_\alpha$ folgt direkt:
			\begin{align*}
				{R^0}_{ijk}&=K_{jk,i}+\inHS\Gamma^m_jk K_{im}-K_{ik,j}-\inHS\Gamma^m_{ik}K_{jm}=\inHS\nabla_i K_{jk}-\inHS\nabla_j K_{ik}\\
				{R^m}_{ijk}&=K_{jk}{K_i}^m-K_{ik}{K_j}^m+\inHS {R^m}_{ijk}
			\end{align*}
		\end{proof}
		Für nähere Informationen zu klassischer Differentialgeometrie von Flächen, siehe \cite{docarmo76}.
		\end{paragraph}
	\section{Retrospektive: Klassische Feldtheorie}
		In der klassischen Feldtheorie übernehmen, die in der klassischen klassischen
		Mechanik mittels Indizes durchnummerierten verallgemeinerten Koordinaten,
		glatte Felder, wobei die Rolle der Indizes durch Raumzeitpunkte abgelöst wird.
		Anstatt der Summe über alle $i$ in der Lagrangefunktion schreibt man ein
		Raumintegral über eine Lagrangedichte, über die wiederum die Wirkung angegeben werden
		kann:
		\begin{align}
			S[\varphi,t]=\int_{t_1}^{t_2}{\rm d}t\int_{\mathbb{R}^3}{\rm d}^3x \mathcal{L}(\varphi,\partial_\mu \varphi,t).
		\end{align}
		Variation des Feldes liefert die Euler-Lagrangegleichung für Felder:
		\begin{align}
			\partial_\mu \frac{\partial \mathcal{L}}{\partial (\partial_\mu \varphi)}-\frac{\partial \mathcal{L}}{\partial \varphi}=0.
		\end{align}
		Analog zum klassischen Fall, kann man die Impulsdichte wie folgt definieren
		\begin{align}
			\pi=\frac{\partial\mathcal{L}}{\partial \dot{\varphi}},
		\end{align}
		was direkt -- über Legendretransformation -- zur Hamiltondichte führt:
		\begin{align}
			\mathcal{H}=\pi\dot{\varphi}-\mathcal{L}
		\end{align}
	\section{ADM-Formalismus}
		Der ADM-Formalismus\cite{2008GReGr..40.1997A, gourgoulhon20123+1} (oft auch 3+1-Formalismus) bezieht sich auf die Formulierung der allgemeinen Relativitätstheorie
		indem die Raumzeit in dreidimensionale raumartige Hyperebenen unterteilt wird. Anders ausgedrückt, lassen sich die
		raumartigen Hyperebenen mittels eines Zeitparameters ``durchwandern'':
		\begin{align}
			M=\bigcup_{t\in\mathbb{R}} \Sigma(t).
		\end{align}
		Eine Bemerkung zu dieser Zerlegung: Um die Existenz zu garantieren, muss vorrausgesetzt werden, dass sich M in jedem
		Punkt zeitordnen lässt. Etwa zeitzyklische Universen, wie etwas das G\"odeluniversum, können mit diesem Formalismus nicht
		abgebildet werden.
		\begin{figure}
			\centering
			
\definecolor{cffffff}{RGB}{255,255,255}


\begin{tikzpicture}[y=0.80pt, x=0.8pt,yscale=-1, inner sep=0pt, outer sep=0pt]
\begin{scope}[shift={(-94.44203,-829.24867)}]
  \path[draw=black,line join=miter,line cap=butt,line width=0.800pt]
    (94.9420,963.6510) .. controls (133.6784,939.3076) and (181.1782,918.7271) ..
    (228.0104,922.6856) .. controls (266.7976,926.8190) and (294.2538,928.1782) ..
    (322.6210,945.4888) .. controls (329.7017,949.9520) and (336.1535,955.4974) ..
    (341.2723,962.1393) .. controls (343.1252,979.8512) and (363.5357,987.5984) ..
    (366.7586,1004.7414) .. controls (311.1900,985.9990) and (249.7531,980.1579)
    .. (198.3765,1002.9182) .. controls (183.7033,1005.2029) and
    (172.5328,991.7686) .. (167.6886,979.4679) .. controls (163.2459,968.0138) and
    (153.0229,955.9553) .. (139.4421,957.8429) .. controls (124.4160,958.9238) and
    (110.3041,967.6749) .. (94.9420,963.7274);
  \path[draw=black,line join=miter,line cap=butt,draw opacity=0.157,line
    width=0.800pt] (169.5537,970.1701) .. controls (184.8251,964.8556) and
    (201.0165,962.5222) .. (217.1456,962.1938);
  \path[draw=black,line join=miter,line cap=butt,draw opacity=0.157,line
    width=0.800pt] (176.6158,982.7480) .. controls (205.2470,978.2455) and
    (233.7414,971.4175) .. (262.8952,971.7040);
  \path[draw=black,line join=miter,line cap=butt,draw opacity=0.157,line
    width=0.800pt] (191.0468,994.0988) .. controls (201.8266,988.2769) and
    (214.6059,989.9168) .. (226.0499,986.4294);
  \path[draw=black,line join=miter,line cap=butt,draw opacity=0.157,line
    width=0.800pt] (174.1594,961.8870) .. controls (182.9650,973.7892) and
    (191.2690,986.3377) .. (202.7145,995.9395);
  \path[draw=black,line join=miter,line cap=butt,draw opacity=0.157,line
    width=0.800pt] (197.4948,958.8192) .. controls (202.6942,971.2479) and
    (208.8145,984.3528) .. (220.5231,991.9514);
  \path[draw=black,line join=miter,line cap=butt,draw opacity=0.157,line
    width=0.800pt] (223.2865,968.6362) .. controls (227.9518,973.3844) and
    (228.9336,980.5503) .. (233.7260,985.2023);
  \path[draw=black,line join=miter,line cap=butt,draw opacity=0.157,line
    width=0.800pt] (242.6302,970.4769) .. controls (245.7097,973.2353) and
    (247.8580,976.8100) .. (249.9993,980.2938);
  \path[draw=black,fill=cffffff,line join=miter,line cap=butt,line width=0.800pt]
    (102.3239,923.6933) .. controls (141.0602,899.3499) and (185.1924,849.7904) ..
    (222.7997,877.9555) .. controls (259.6328,906.8185) and (282.3126,886.7020) ..
    (310.6798,904.0126) .. controls (325.6340,918.2228) and (365.7555,951.6900) ..
    (367.1928,958.0590) .. controls (327.6285,907.0611) and (221.1669,910.0969) ..
    (173.5506,951.8749) .. controls (168.8908,942.3732) and (147.5951,935.0871) ..
    (134.0143,936.9747) .. controls (118.9882,938.0556) and (117.6859,927.5003) ..
    (102.3239,923.5528);
  \path[draw=black,fill=cffffff,line join=miter,line cap=butt,line width=0.800pt]
    (98.2206,864.5189) .. controls (136.9569,840.1754) and (170.6676,898.6450) ..
    (208.2749,852.8384) .. controls (238.7467,817.3350) and (289.6116,830.9577) ..
    (325.3480,840.2921) .. controls (324.3358,865.8531) and (362.0864,906.3988) ..
    (363.5238,912.7678) .. controls (343.7416,887.2688) and (285.3067,853.3688) ..
    (269.7221,874.8848) .. controls (247.7278,905.2499) and (193.6897,885.6947) ..
    (169.8815,906.5837) .. controls (165.2217,897.0819) and (141.5377,902.8114) ..
    (127.9569,904.6990) .. controls (112.9309,905.7799) and (113.3655,868.5427) ..
    (98.0034,864.5953);
  \path[draw=black,line join=miter,line cap=butt,draw opacity=0.157,line
    width=0.800pt] (172.9312,933.6634) .. controls (182.9969,928.4867) and
    (193.4458,923.8070) .. (204.5568,921.3922)(158.8072,928.7549) .. controls
    (175.9207,921.0337) and (194.5262,917.6959) ..
    (212.5399,912.8024)(207.9343,910.6549) .. controls (210.4159,911.3053) and
    (211.5694,913.8430) .. (213.7681,914.9498)(189.5116,913.1091) .. controls
    (193.7087,914.8489) and (197.4618,917.5019) ..
    (200.8722,920.4718)(170.4749,918.6312) .. controls (177.3657,919.7637) and
    (183.4697,923.4371) .. (189.2046,927.2210)(146.5254,925.0735) .. controls
    (157.9106,923.1662) and (169.4885,927.0688) ..
    (179.0721,933.0498)(138.6958,898.6905) .. controls (146.8266,898.2313) and
    (159.1248,886.7016) .. (166.4833,884.2719)(125.6464,888.2600) .. controls
    (137.5567,888.0534) and (148.6698,883.2307) ..
    (159.7283,879.3634)(144.9902,878.7499) .. controls (145.7319,888.9832) and
    (158.8597,882.9274) .. (163.7199,888.8736)(125.3394,877.5227) .. controls
    (126.0847,893.3290) and (145.6004,895.0514) ..
    (157.5790,893.7820)(230.6555,877.2160) .. controls (239.4159,864.6157) and
    (253.1288,856.2295) .. (267.5008,851.4465)(266.5797,841.0160) .. controls
    (246.4317,843.5436) and (228.3279,855.2617) ..
    (215.3033,870.4668)(255.8331,836.7211) .. controls (261.1550,844.6456) and
    (264.8590,853.4855) .. (269.3431,861.8770)(227.5851,848.3787) .. controls
    (238.6699,849.5924) and (247.5544,857.8010) ..
    (253.3768,866.7855)(203.9427,865.2516) .. controls (220.4516,861.9635) and
    (236.8489,868.9342) .. (250.9204,876.9092);
  \path[draw=black,fill=black,line width=0.515pt] (204.5153,945.5931) --
    (204.8155,951.9510) -- (203.0985,950.3631) -- (200.7924,950.7581) -- cycle;
  \path[draw=black,line join=miter,line cap=butt,line width=0.800pt]
    (203.0901,950.3516) -- (199.2063,963.4272);
  \path[fill=black] (209.69363,951.6225) node[above right] (text4047) {$n$};
  \path[draw=black,fill=black,line width=0.515pt] (357.0898,942.3206) --
    (362.2586,938.6001) -- (361.8637,940.9042) -- (363.4533,942.6194) -- cycle;
  \path[draw=black,line join=miter,line cap=butt,line width=0.800pt]
    (361.8586,940.9174) .. controls (366.2081,939.6189) and (380.0721,935.4876) ..
    (387.7544,935.2991);
  \path[draw=black,fill=black,line width=0.515pt] (364.5406,991.5995) --
    (368.1648,986.3649) -- (368.5645,988.6682) -- (370.6370,989.7529) -- cycle;
  \path[draw=black,line join=miter,line cap=butt,line width=0.800pt]
    (368.5642,988.6823) .. controls (390.0560,974.3016) and (382.0742,976.0271) ..
    (392.8386,958.5207);
  \path[fill=black] (398.88522,964.6604) node[above right] (text4063)
    {$\Sigma(t)$};
  \path[fill=black] (389.55313,938.03009) node[above right] (text4067)
    {$\Sigma(t+\rm{d}t)$};
\end{scope}

\end{tikzpicture}

			\caption{Schematische Darstellung der Zerlegung der Raumzeit in Hyperebenen}
		\end{figure}
		
		Zur Beschreibung der dynamischen Phänomene ist es zuerst notwendig eine günstige Darstellung der Metrik
		zu finden. Dazu wählt man die Länge des Linienelements $ds$ wie folgt:
		\begin{align}
			{\rm d}s^2=&-{\rm (Zeitartiger\ Abstand)}^2+{\rm (Raumartiger\ Abstand)}^2=\notag\\
			=&-N^2 {\rm d}t^2+h_{ij}({\rm d}x^i+N^i {\rm d}t)({\rm d}x^j+N^j {\rm d}t).
		\end{align}
		Als Matrix lässt sich die allgemeine Metric $g$ mittels $N,\ N^i$ und $h_{ij}$ schreiben als
		\begin{align}
			[g]=\left(
			\begin{array}{cccc}
				-N^2+h_{ij}N^i N^j & h_{1j}N^j & h_{2j}N^j & h_{3j}N^j \\
				h_{1j}N^j & h_{11} & h_{12} & h_{13}\\
				h_{2j}N^j & h_{21} & h_{22} & h_{23}\\
				h_{3j}N^j & h_{31} & h_{32} & h_{33}
			\end{array}\right)
			\label{equ:gmatrix}
		\end{align}
	\subsection{Einstein-Hilbert Wirkung und super-Hamilton constraints}
		Wie Hilbert zeigen konnte, lässt sich für die einstein'schen Feldgleichungen eine Lagrangedichte
		angeben, womit die Wirkung angegeben werden kann:
		\begin{align}
			S_{EH}=\frac{1}{16\pi G}\int_M{\rm d}^4x \sqrt{-g}R
		\end{align}
		wobei $g= {\rm det}(g)$ und $R$ den Ricci-Skalar bezeichnet.
		$S_{EH}$ lässt sich nun als Funktional auf Lagrangefunktionen mit den Argumenten $N,N^i$ und $h_{ij}$ umschreiben.
		
		Der aufmerksame Beobachter, stellt fest, dass die erste Spalte der Matrixdarstellung von g aus (\ref{equ:gmatrix}) aus 
		Summen vielfachen der anderen 3 Spalten geschrieben werden kann, nämlich -- $[g_i]$ sei die $i$-te Spalte, wobei von
		0 begonnen wird zu zählen -- wie folgt:
		\begin{align}
			[g_0]=\left(\begin{array}{c}-N^2\\0\\0\\0\end{array}\right)+N^i[g_i].
		\end{align}
		Aufgrund elementarer Eigenschaften der Determinante gilt offensichtlich:
		\begin{align}
			g={\rm det}(g)={\rm det}\left(
			\begin{array}{cccc}
				-N^2 & h_{1j}N^j & h_{2j}N^j & h_{3j}N^j \\
				0 & h_{11} & h_{12} & h_{13}\\
				0 & h_{21} & h_{22} & h_{23}\\
				0 & h_{31} & h_{32} & h_{33}
			\end{array}\right)=-N^2 {\rm det}(h).
		\end{align}
		Als nächsten Schritt gilt es, den Ricci-Skalar in den neuen Argumenten auszudrücken. Aus den Gauss-Codazzi Gleichungen folgt unmittelbar
		\begin{align*}
			R=\inHS R+{\rm Tr}(K)^2-K^{ij}K_{ij}.
		\end{align*}
		Die Lagrangedichte ist somit $\mathcal{L}=\frac{1}{16\pi G} N\sqrt{h}(\inHS R+{\rm Tr}(K)^2-K^{ij}K_{ij})$. 
		
		Daraus ergibt sich mit
		\begin{align}
			\pi_{ij}:=\frac{\partial \mathcal{L}}{\partial(\partial_t h_{ij})}.
		\end{align}
		Ignorieren der Randterme führt schließlich zum Ausdruck
		\begin{align}
			S_{EH}=\int{\rm d}t\int_\Sigma {\rm d}^3x (\pi^{ij}\partial_t h_{ij}-NH_{\perp}-N^i H_i).
		\end{align}
		Man erkennt sofort, dass diese Funktion nicht abhängig von $\partial_t N$ und $\partial_t N^i$
		ist, womit die Ableitungen der der Hamiltonfunktion nach $N$ und $N^i$ konstant entlang der gewählten
		Zeitentwicklung bleiben. Konkret sind die Größen
		\begin{align}
			&H_{\perp}:=16\pi G G_{ijkl}\pi^{ij}\pi^{kl}-\frac{1}{16\pi G}\sqrt{h}\inHS R=0 \label{equ:constraint1}\\
			&H^i:=-2\inHS\nabla_j\pi^{ij}=0 \label{equ:constraint2}
		\end{align}
		zeitinvariant. Erstere wird super-hamiltonian constraint, mit der super-DeWitt Metrik $G_{ijkl}=\frac{1}{2\sqrt{h}}(h_{ik}h_{jl}+h_{jk}h_{il}-h_{ij}h_{kl})$,
		und zweitere super-momentum constraint genannt.
		Mit der üblichen Poissonklammern für $h_{ij}$ und $\pi_{ij}$,
		\begin{align}
			\{h_{ij}(\vec{x}),\pi^{kl}(\vec{x}^{\,\prime})\}=\delta^k_{(i}\delta^l_{j)}\delta(\vec{x}-\vec{x}^{\,\prime}),
		\end{align}
		kann gezeigt werden \cite{dirac2001lectures,cjm19510012}, dass die Zwangsbedingungen die Kommutatorrelationen \cite{qg06}
		\begin{align}
			&\{H_i(\vec{x}),H_j(\vec{x}^{\,\prime})\}=H_i(\vec{x}^{\,\prime})\partial_j\delta(\vec{x}-\vec{x}^{\,\prime})-H_j(\vec{x})\partial\vec{x}^{\,\prime}_i\delta(\vec{x}-\vec{x}^{\,\prime})\\
			&\{H_i(\vec{x}),H_\perp(\vec{x}^{\,\prime})\}=H_\perp(\vec{x})\partial_j\delta(\vec{x}-\vec{x}^{\,\prime})\\
			&\{H_i(\vec{x}),H_\perp(\vec{x}^{\,\prime})\}=h^{ij}(\vec{x})H_i(\vec{x})\partial^{\,\prime}_j\delta(\vec{x}-\vec{x}^{\,\prime})-h^{ij}(\vec{x}^{\,\prime})H_i(\vec{x}^{\,\prime})\partial_j\delta(\vec{x}-\vec{x}^{\,\prime})
		\end{align}
		erfüllen.
	\subsection{Tieferliegende Rolle der Zwangsbedingungen}
		Es stellt sich heraus, dass die Zwangsbedingungen eine äußerst wichtige Rolle in der
		Entwicklung eines Systems spielen. Tatsächlich enthalten sie die selbe dynamische
		Information wie die einsteinschen Feldgleichungen, wie folgender
		Satz zeigt:
		\begin{theorem}
			Eine Lorentzmetrik $h$ erf\"ullt die einsteinschen Feldgleichungen genau dann, wenn die
			Zwangsbedingungen (\ref{equ:constraint1}-\ref{equ:constraint2}) erf\"ullt sind.
			\label{thm:full}
		\end{theorem}
		\begin{proof}
			Siehe \cite{gr-qc/9210011}. % TODO
		\end{proof}
	\section{Kanonische Quantisierung}
	\subsection{Allgemeine Wiederholung}
		Bei der kanonischen Quantisierung geht man von einer Lagrangedichte für ein 
		Feld aus, und erhebt die klassischen reelen Funktionen $\varphi(x)$ und $\pi(x)$ in
		den Rang von Operatoren.\\
		Ohne zu tief in die Materie eintauchen zu wollen, ist es als Beispiel aber nichtsdestoweniger
		hilfreich sich den Fall eines freien massereichen skalaren Feldes zu Gemüte zu führen.
		
		Die Langrangedichte für ein solches Feld lautet
		\begin{align}
			\mathcal{L}=\frac{1}{2}(\partial_\mu\varphi \partial^\mu\varphi-m^2\varphi^2)
		\end{align}
		
		Um mit den Feldoperatoren in einem vernünftigen Rahmen rechnen zu können
		ist es notwendig zu wissen wie, verschiedene Operatoren mitteinander in 
		Beziehung stehen. Das gelingt indem man Kommutatorrelationen angiebt.
		Diese verallgemeinern sich auf natürliche (kanonische) Weise zu
		\begin{align}
			&[\hat{\varphi}(\vec{x}), \hat{\varphi}(\vec{y})]=0=[\hat{\pi}(\vec{x}), \hat{\pi}(\vec{y})] \\
			&[\hat{\varphi}(\vec{x}), \hat{\pi}(\vec{y})]=i\delta(\vec{x}-\vec{y}).
		\end{align}
		Eine Bemerkung zu Fällen, in denen die Ableitung der Felder auftritt:
		Hier ist es natürlich notwendig das Konzept der Ableitung auch entsprechend
		zu verallgemeinern. Das wird mithilfe der funktionalen Ableitung erreicht
		(siehe Fr\'echet Ableitung bzw. G\^ateaux Ableitung).
		Beispielsweise wirkt der Impulsoperator auf Wellenfunktionen in Ortsdarstellung in quantisierten Feldtheorien folgendermaßen:
		\begin{align*}
			\hat{\pi}\Psi[\varphi]=-i\frac{\delta\Psi[\varphi]}{\delta \varphi}.
		\end{align*}
		Falls keine Felder beschrieben werden, reduziert sich diese Gleichung auf die altbekannte
		\begin{align*}
			\vec{p}\, \Psi(\vec{x},t)=-i\nabla \Psi(\vec{x},t),
		\end{align*}
		wobei $\nabla$ hier natürlich den üblichen Nablaoperator und nicht eine kovariante Ableitung darstellt.
	\subsection{Kanonische Quantisierung des Hamiltonformalismus der Gravitation}
		Analog zu dem vorherigen Abschnitt werden nun die Feldgrößen zu Operatoren.
		Die Poissonklammerausdrücke wird weiters durch den Kommutatoren ersetzt.
		Folgt man den Spuren Dirac's bis hin zu seiner Quantisierung von Zwangsbedingungen \cite{dirac2001lectures}
		erhällt man die Funktionaldiffernetialgleichungen, welche zusammen mit Satz \ref{thm:full} die Vollständige
		Information über das Verhalten der Raumzeit enthalten. Die quantisierten Constraints wirken auf Wellenfunktionen folgendermaßen:
		\begin{align}
			\hat{H}_\perp\Psi[h]&=0\\
			\hat{H}_i\Psi[h]&=0.
		\end{align}
		Ausgeschrieben erhällt man beispielsweise für die Gleichung die sich aus dem Hamilton constraint ergibt:
		\begin{align}
			-16\pi G G_{ijkl}\frac{\delta^2\Psi[h]}{\delta h_{ij}\delta h_{kl}}-\frac{1}{16\pi G}\inHS R\Psi[h]=0.
		\end{align}
		Dieses Result ist als Wheeler-DeWitt Gleichung bekannt und steht hier in allgemeiner Form. Im nächsten Kapitel
		wird diese Gleichung für ein stark vereinfachtes System betrachtet in dem sie sich auf eine gewöhnliche 
		Differentialgleichung reduziert.
		
		Man erkennt außerdem, dass in dieser Gleichung die Zeit nicht explizit vorkommen. Dies ist ein erstes Indiz dafür, dass
		die Zeit in der Quantengravitation eine andere Rolle einnimmt, als in den bisher entwickelten Quantenfeldtheorien.
		Diese Rolle ist noch nicht eruiert und aktuelles Forschungsgebiet. Es existieren hierzu jedoch diverse
		Interpretationsansätze, die das Problem unterschiedlich beleuchten.
	\section{Problem der Zeit in der Quantengravitation}
		Um nicht zu weit Abzuschweifen, muss in diesem Kapitel auf tieferliegende
		mathematische Begründungen leider verzichtet werden. Für weiterführende Referenzen
		siehe \cite{qg06,gr-qc/9210011}.
		
		Wie bereits in vorherigem Abschnitt beobachtet, kommen in der Wheeler-DeWitt Gleichung keine
		expliziten Zeitabhängigkeiten vor. Natürlich ist das nicht überraschend, da bereits in der klassischen Version
		offensichtlich keine Zeit in den Zwangsbedingungen auftritt -- es wurden ja nur Funktionen
		mit Operatoren ersetzt -- führt aber dennoch zur konzeptuell unterschiedlichen Zeitinterpretationen, nämlich,
		dass man der Zeit entweder vor oder nach der Quantisierung Bedeutung schenkt. Diese Entscheidungsfreiheit charakterisiert
		die ersten beiden Auslegungen. In der dritte der gängigen Anschauungen auf das Problem der Zeit 
		wird die Zeit von ihrer bisherigen Sonderstellung enttront und es wird ihr nur noch die Bedeutung
		einer phänomenologischen Erscheinung zugeteilt. Dies manifestiert sich in ansich zeitlosen Theorien.
		
		In den folgenden Unterabschnitten werden Beispiele für diese Interpretationen genannt
		und die darin auftretenden Vor- und Nachteile kurz besprochen.
		\subsection{Identifikation der Zeit vor der Quantisierung} % matter clocks and reference fluids
			Die erste Möglichkeit, das Problem der Zeit zu behandeln ist die Identifikation der
			Zeit als Funktional von $h_{ij}$ und $\pi_{ij}$ vor der Quantisierung. Eine der Realisierungen
			ist beispielsweise die Implementierung des Zeitbegriffs via ,,Matter Clocks`` und ,,Reference Fluids`` \cite{PhysRevD.43.419,gr-qc/9210011}. %% TODO %%
		\subsection{Identifikation der Zeit nach der Quantisierung} % cannonical quantum gravity & semiclassical approaches (WKB)
			Alternativ lässt sich die Zeit auch nach der Quantisierung ,,rekonstruieren``. Prominente 
			Vertreter dieser Herangehensweise können für einfache Modelle zu Formen der Wheeler-DeWitt Gleichung
			führen, welche stark an die Schrödingergleichung erinnern.
			
			Ein solches Bespiel \cite{PhysRevD.28.2960} lässt sich mit einem Minisuperspace, was ein stark vereinfachtes System 
			beschreibt, verdeutlichen. In diesem Modell wird ein expandierendes Universum mit Topologie
			$\mathbb{R}\times \mathcal{S}^3$ beschrieben, wobei $\mathcal{S}^3$ eine 3-Sphäre bezeichnet (Eine Kugel im $\mathbb{R}^4$ mit dreidimensionaler Oberfläche).
			\begin{figure}
				\centering
				
\definecolor{cffffff}{RGB}{255,255,255}
\definecolor{c2f2f2f}{RGB}{47,47,47}
\definecolor{c5d5d5d}{RGB}{93,93,93}
\definecolor{c858585}{RGB}{133,133,133}
\definecolor{ca0a0a0}{RGB}{160,160,160}
\definecolor{cbbbbbb}{RGB}{187,187,187}
\definecolor{ccccccc}{RGB}{204,204,204}
\definecolor{ce1e1e1}{RGB}{225,225,225}
\definecolor{c000003}{RGB}{0,0,3}


\begin{tikzpicture}[y=0.80pt, x=0.8pt,yscale=-1, inner sep=0pt, outer sep=0pt]
  \begin{scope}[fill=cffffff,fill opacity=0.810,transparency group]
    \path[draw=black,fill=cffffff,line join=miter,line cap=butt,fill
      opacity=0.810,even odd rule,line width=0.800pt] (214.2812,740.9375) ..
      controls (140.1173,740.9375) and (80.0000,801.0548) .. (80.0000,875.2188) ..
      controls (80.0000,949.3827) and (140.1173,1009.5000) .. (214.2812,1009.5000)
      .. controls (288.4452,1009.5000) and (348.5625,949.3827) ..
      (348.5625,875.2188) .. controls (348.5625,801.0548) and (288.4452,740.9375) ..
      (214.2812,740.9375) -- cycle(244.5000,790.8438) .. controls
      (268.0981,791.3420) and (273.2571,821.4646) .. (288.2500,835.3125) .. controls
      (301.2515,856.3298) and (325.2448,871.8489) .. (328.5000,897.8438) .. controls
      (327.9493,921.0561) and (298.4384,924.0762) .. (290.2188,943.0000) .. controls
      (286.2271,963.2847) and (261.1112,968.6710) .. (251.7812,948.8750) .. controls
      (243.4201,932.7436) and (228.3908,913.1423) .. (241.7500,895.3750) .. controls
      (258.0833,877.6621) and (245.0726,855.1831) .. (231.8125,840.2812) .. controls
      (217.5359,827.1128) and (216.1269,796.0533) .. (239.1562,792.0000) --
      (241.7812,791.2500) -- (244.5000,790.8438) -- cycle;
    \path[draw=black,fill=cffffff,line join=miter,line cap=butt,fill
      opacity=0.810,line width=0.800pt] (236.7177,908.3244) .. controls
      (229.4232,908.6311) and (221.9290,908.7907) .. (214.2857,908.7907) .. controls
      (177.2037,908.7907) and (143.6323,905.0331) .. (119.3314,898.9579) .. controls
      (95.0304,892.8827) and (80.0000,884.4898) .. (80.0000,875.2193);
    \path[draw=black,fill=cffffff,line join=miter,line cap=butt,fill
      opacity=0.810,line width=0.800pt] (348.5714,875.2193) .. controls
      (348.5714,881.8608) and (340.8571,888.0519) .. (327.5463,893.2630);
  \end{scope}
    \path[draw=black,fill=cffffff,line join=miter,line cap=butt,even odd rule,line
      width=0.800pt] (214.2812,740.9375) .. controls (140.1173,740.9375) and
      (80.0000,801.0548) .. (80.0000,875.2188) .. controls (80.0000,949.3827) and
      (140.1173,1009.5000) .. (214.2812,1009.5000) .. controls (288.4452,1009.5000)
      and (348.5625,949.3827) .. (348.5625,875.2188) .. controls (348.5625,801.0548)
      and (288.4452,740.9375) .. (214.2812,740.9375) -- cycle(244.5000,790.8438) ..
      controls (268.0981,791.3420) and (273.2571,821.4646) .. (288.2500,835.3125) ..
      controls (301.2515,856.3298) and (325.2448,871.8489) .. (328.5000,897.8438) ..
      controls (327.9493,921.0561) and (298.4384,924.0762) .. (290.2188,943.0000) ..
      controls (286.2271,963.2847) and (261.1112,968.6710) .. (251.7812,948.8750) ..
      controls (243.4201,932.7436) and (228.3908,913.1423) .. (241.7500,895.3750) ..
      controls (258.0833,877.6621) and (245.0726,855.1831) .. (231.8125,840.2812) ..
      controls (217.5359,827.1128) and (216.1269,796.0533) .. (239.1562,792.0000) --
      (241.7812,791.2500) -- (244.5000,790.8438) -- cycle;
    \path[draw=black,line join=miter,line cap=butt,line width=0.800pt]
      (236.7177,908.3244) .. controls (229.4232,908.6311) and (221.9290,908.7907) ..
      (214.2857,908.7907) .. controls (177.2037,908.7907) and (143.6323,905.0331) ..
      (119.3314,898.9579) .. controls (95.0304,892.8827) and (80.0000,884.4898) ..
      (80.0000,875.2193);
    \path[draw=black,line join=miter,line cap=butt,line width=0.800pt]
      (348.5714,875.2193) .. controls (348.5714,881.8608) and (340.8571,888.0519) ..
      (327.5463,893.2630);
  \path (238.9011,904.6274)arc(0.000:180.000:33.713840 and
    39.143)arc(-180.000:0.000:33.713840 and 39.143) -- cycle;
  \begin{scope}[fill=cffffff,fill opacity=0.810,transparency group]
    \path[cm={{2.57928,0.0,0.0,2.57928,(-372.50682,-1397.1565)}},draw=black,fill=black]
      (261.3770,884.6769)arc(-0.016:180.016:25.506)arc(-180.016:0.016:25.506) --
      cycle;
    \begin{scope}[cm={{0.65372,0.0,0.0,0.65372,(74.20306,303.07158)}},draw=black,fill=c2f2f2f]
      \path[draw=black,fill=c2f2f2f,line join=miter,line cap=butt,even odd rule,line
        width=0.800pt] (214.2812,740.9375) .. controls (140.1173,740.9375) and
        (80.0000,801.0548) .. (80.0000,875.2188) .. controls (80.0000,949.3827) and
        (140.1173,1009.5000) .. (214.2812,1009.5000) .. controls (288.4452,1009.5000)
        and (348.5625,949.3827) .. (348.5625,875.2188) .. controls (348.5625,801.0548)
        and (288.4452,740.9375) .. (214.2812,740.9375) -- cycle(244.5000,790.8438) ..
        controls (268.0981,791.3420) and (273.2571,821.4646) .. (288.2500,835.3125) ..
        controls (301.2515,856.3298) and (325.2448,871.8489) .. (328.5000,897.8438) ..
        controls (327.9493,921.0561) and (298.4384,924.0762) .. (290.2188,943.0000) ..
        controls (286.2271,963.2847) and (261.1112,968.6710) .. (251.7812,948.8750) ..
        controls (243.4201,932.7436) and (228.3908,913.1423) .. (241.7500,895.3750) ..
        controls (258.0833,877.6621) and (245.0726,855.1831) .. (231.8125,840.2812) ..
        controls (217.5359,827.1128) and (216.1269,796.0533) .. (239.1562,792.0000) --
        (241.7812,791.2500) -- (244.5000,790.8438) -- cycle;
      \path[draw=black,fill=c2f2f2f,line join=miter,line cap=butt,line width=0.800pt]
        (236.7177,908.3244) .. controls (229.4232,908.6311) and (221.9290,908.7907) ..
        (214.2857,908.7907) .. controls (177.2037,908.7907) and (143.6323,905.0331) ..
        (119.3314,898.9579) .. controls (95.0304,892.8827) and (80.0000,884.4898) ..
        (80.0000,875.2193);
      \path[draw=black,fill=c2f2f2f,line join=miter,line cap=butt,line width=0.800pt]
        (348.5714,875.2193) .. controls (348.5714,881.8608) and (340.8571,888.0519) ..
        (327.5463,893.2630);
    \end{scope}
    \begin{scope}[cm={{0.67128,0.0,0.0,0.67128,(70.44035,287.70334)}},draw=black,fill=c5d5d5d]
      \path[draw=black,fill=c5d5d5d,line join=miter,line cap=butt,even odd rule,line
        width=0.800pt] (214.2812,740.9375) .. controls (140.1173,740.9375) and
        (80.0000,801.0548) .. (80.0000,875.2188) .. controls (80.0000,949.3827) and
        (140.1173,1009.5000) .. (214.2812,1009.5000) .. controls (288.4452,1009.5000)
        and (348.5625,949.3827) .. (348.5625,875.2188) .. controls (348.5625,801.0548)
        and (288.4452,740.9375) .. (214.2812,740.9375) -- cycle(244.5000,790.8438) ..
        controls (268.0981,791.3420) and (273.2571,821.4646) .. (288.2500,835.3125) ..
        controls (301.2515,856.3298) and (325.2448,871.8489) .. (328.5000,897.8438) ..
        controls (327.9493,921.0561) and (298.4384,924.0762) .. (290.2188,943.0000) ..
        controls (286.2271,963.2847) and (261.1112,968.6710) .. (251.7812,948.8750) ..
        controls (243.4201,932.7436) and (228.3908,913.1423) .. (241.7500,895.3750) ..
        controls (258.0833,877.6621) and (245.0726,855.1831) .. (231.8125,840.2812) ..
        controls (217.5359,827.1128) and (216.1269,796.0533) .. (239.1562,792.0000) --
        (241.7812,791.2500) -- (244.5000,790.8438) -- cycle;
      \path[draw=black,fill=c5d5d5d,line join=miter,line cap=butt,line width=0.800pt]
        (236.7177,908.3244) .. controls (229.4232,908.6311) and (221.9290,908.7907) ..
        (214.2857,908.7907) .. controls (177.2037,908.7907) and (143.6323,905.0331) ..
        (119.3314,898.9579) .. controls (95.0304,892.8827) and (80.0000,884.4898) ..
        (80.0000,875.2193);
      \path[draw=black,fill=c5d5d5d,line join=miter,line cap=butt,line width=0.800pt]
        (348.5714,875.2193) .. controls (348.5714,881.8608) and (340.8571,888.0519) ..
        (327.5463,893.2630);
    \end{scope}
    \begin{scope}[cm={{0.69103,0.0,0.0,0.69103,(66.2073,270.41406)}},draw=black,fill=c858585]
      \path[draw=black,fill=c858585,line join=miter,line cap=butt,even odd rule,line
        width=0.800pt] (214.2812,740.9375) .. controls (140.1173,740.9375) and
        (80.0000,801.0548) .. (80.0000,875.2188) .. controls (80.0000,949.3827) and
        (140.1173,1009.5000) .. (214.2812,1009.5000) .. controls (288.4452,1009.5000)
        and (348.5625,949.3827) .. (348.5625,875.2188) .. controls (348.5625,801.0548)
        and (288.4452,740.9375) .. (214.2812,740.9375) -- cycle(244.5000,790.8438) ..
        controls (268.0981,791.3420) and (273.2571,821.4646) .. (288.2500,835.3125) ..
        controls (301.2515,856.3298) and (325.2448,871.8489) .. (328.5000,897.8438) ..
        controls (327.9493,921.0561) and (298.4384,924.0762) .. (290.2188,943.0000) ..
        controls (286.2271,963.2847) and (261.1112,968.6710) .. (251.7812,948.8750) ..
        controls (243.4201,932.7436) and (228.3908,913.1423) .. (241.7500,895.3750) ..
        controls (258.0833,877.6621) and (245.0726,855.1831) .. (231.8125,840.2812) ..
        controls (217.5359,827.1128) and (216.1269,796.0533) .. (239.1562,792.0000) --
        (241.7812,791.2500) -- (244.5000,790.8438) -- cycle;
      \path[draw=black,fill=c858585,line join=miter,line cap=butt,line width=0.800pt]
        (236.7177,908.3244) .. controls (229.4232,908.6311) and (221.9290,908.7907) ..
        (214.2857,908.7907) .. controls (177.2037,908.7907) and (143.6323,905.0331) ..
        (119.3314,898.9579) .. controls (95.0304,892.8827) and (80.0000,884.4898) ..
        (80.0000,875.2193);
      \path[draw=black,fill=c858585,line join=miter,line cap=butt,line width=0.800pt]
        (348.5714,875.2193) .. controls (348.5714,881.8608) and (340.8571,888.0519) ..
        (327.5463,893.2630);
    \end{scope}
    \begin{scope}[cm={{0.71252,0.0,0.0,0.71252,(61.60214,251.60495)}},draw=black,fill=ca0a0a0]
      \path[draw=black,fill=ca0a0a0,line join=miter,line cap=butt,even odd rule,line
        width=0.800pt] (214.2812,740.9375) .. controls (140.1173,740.9375) and
        (80.0000,801.0548) .. (80.0000,875.2188) .. controls (80.0000,949.3827) and
        (140.1173,1009.5000) .. (214.2812,1009.5000) .. controls (288.4452,1009.5000)
        and (348.5625,949.3827) .. (348.5625,875.2188) .. controls (348.5625,801.0548)
        and (288.4452,740.9375) .. (214.2812,740.9375) -- cycle(244.5000,790.8438) ..
        controls (268.0981,791.3420) and (273.2571,821.4646) .. (288.2500,835.3125) ..
        controls (301.2515,856.3298) and (325.2448,871.8489) .. (328.5000,897.8438) ..
        controls (327.9493,921.0561) and (298.4384,924.0762) .. (290.2188,943.0000) ..
        controls (286.2271,963.2847) and (261.1112,968.6710) .. (251.7812,948.8750) ..
        controls (243.4201,932.7436) and (228.3908,913.1423) .. (241.7500,895.3750) ..
        controls (258.0833,877.6621) and (245.0726,855.1831) .. (231.8125,840.2812) ..
        controls (217.5359,827.1128) and (216.1269,796.0533) .. (239.1562,792.0000) --
        (241.7812,791.2500) -- (244.5000,790.8438) -- cycle;
      \path[draw=black,fill=ca0a0a0,line join=miter,line cap=butt,line width=0.800pt]
        (236.7177,908.3244) .. controls (229.4232,908.6311) and (221.9290,908.7907) ..
        (214.2857,908.7907) .. controls (177.2037,908.7907) and (143.6323,905.0331) ..
        (119.3314,898.9579) .. controls (95.0304,892.8827) and (80.0000,884.4898) ..
        (80.0000,875.2193);
      \path[draw=black,fill=ca0a0a0,line join=miter,line cap=butt,line width=0.800pt]
        (348.5714,875.2193) .. controls (348.5714,881.8608) and (340.8571,888.0519) ..
        (327.5463,893.2630);
    \end{scope}
    \begin{scope}[cm={{0.74772,0.0,0.0,0.74772,(54.05893,220.79581)}},draw=black,fill=cbbbbbb]
      \path[draw=black,fill=cbbbbbb,line join=miter,line cap=butt,even odd rule,line
        width=0.800pt] (214.2812,740.9375) .. controls (140.1173,740.9375) and
        (80.0000,801.0548) .. (80.0000,875.2188) .. controls (80.0000,949.3827) and
        (140.1173,1009.5000) .. (214.2812,1009.5000) .. controls (288.4452,1009.5000)
        and (348.5625,949.3827) .. (348.5625,875.2188) .. controls (348.5625,801.0548)
        and (288.4452,740.9375) .. (214.2812,740.9375) -- cycle(244.5000,790.8438) ..
        controls (268.0981,791.3420) and (273.2571,821.4646) .. (288.2500,835.3125) ..
        controls (301.2515,856.3298) and (325.2448,871.8489) .. (328.5000,897.8438) ..
        controls (327.9493,921.0561) and (298.4384,924.0762) .. (290.2188,943.0000) ..
        controls (286.2271,963.2847) and (261.1112,968.6710) .. (251.7812,948.8750) ..
        controls (243.4201,932.7436) and (228.3908,913.1423) .. (241.7500,895.3750) ..
        controls (258.0833,877.6621) and (245.0726,855.1831) .. (231.8125,840.2812) ..
        controls (217.5359,827.1128) and (216.1269,796.0533) .. (239.1562,792.0000) --
        (241.7812,791.2500) -- (244.5000,790.8438) -- cycle;
      \path[draw=black,fill=cbbbbbb,line join=miter,line cap=butt,line width=0.800pt]
        (236.7177,908.3244) .. controls (229.4232,908.6311) and (221.9290,908.7907) ..
        (214.2857,908.7907) .. controls (177.2037,908.7907) and (143.6323,905.0331) ..
        (119.3314,898.9579) .. controls (95.0304,892.8827) and (80.0000,884.4898) ..
        (80.0000,875.2193);
      \path[draw=black,fill=cbbbbbb,line join=miter,line cap=butt,line width=0.800pt]
        (348.5714,875.2193) .. controls (348.5714,881.8608) and (340.8571,888.0519) ..
        (327.5463,893.2630);
    \end{scope}
    \begin{scope}[cm={{0.80102,0.0,0.0,0.80102,(42.63925,174.15381)}},draw=black,fill=ccccccc]
      \path[draw=black,fill=ccccccc,line join=miter,line cap=butt,even odd rule,line
        width=0.800pt] (214.2812,740.9375) .. controls (140.1173,740.9375) and
        (80.0000,801.0548) .. (80.0000,875.2188) .. controls (80.0000,949.3827) and
        (140.1173,1009.5000) .. (214.2812,1009.5000) .. controls (288.4452,1009.5000)
        and (348.5625,949.3827) .. (348.5625,875.2188) .. controls (348.5625,801.0548)
        and (288.4452,740.9375) .. (214.2812,740.9375) -- cycle(244.5000,790.8438) ..
        controls (268.0981,791.3420) and (273.2571,821.4646) .. (288.2500,835.3125) ..
        controls (301.2515,856.3298) and (325.2448,871.8489) .. (328.5000,897.8438) ..
        controls (327.9493,921.0561) and (298.4384,924.0762) .. (290.2188,943.0000) ..
        controls (286.2271,963.2847) and (261.1112,968.6710) .. (251.7812,948.8750) ..
        controls (243.4201,932.7436) and (228.3908,913.1423) .. (241.7500,895.3750) ..
        controls (258.0833,877.6621) and (245.0726,855.1831) .. (231.8125,840.2812) ..
        controls (217.5359,827.1128) and (216.1269,796.0533) .. (239.1562,792.0000) --
        (241.7812,791.2500) -- (244.5000,790.8438) -- cycle;
      \path[draw=black,fill=ccccccc,line join=miter,line cap=butt,line width=0.800pt]
        (236.7177,908.3244) .. controls (229.4232,908.6311) and (221.9290,908.7907) ..
        (214.2857,908.7907) .. controls (177.2037,908.7907) and (143.6323,905.0331) ..
        (119.3314,898.9579) .. controls (95.0304,892.8827) and (80.0000,884.4898) ..
        (80.0000,875.2193);
      \path[draw=black,fill=ccccccc,line join=miter,line cap=butt,line width=0.800pt]
        (348.5714,875.2193) .. controls (348.5714,881.8608) and (340.8571,888.0519) ..
        (327.5463,893.2630);
    \end{scope}
    \begin{scope}[cm={{0.87259,0.0,0.0,0.87259,(27.30232,111.51236)}},draw=black,fill=ce1e1e1]
      \path[draw=black,fill=ce1e1e1,line join=miter,line cap=butt,even odd rule,line
        width=0.800pt] (214.2812,740.9375) .. controls (140.1173,740.9375) and
        (80.0000,801.0548) .. (80.0000,875.2188) .. controls (80.0000,949.3827) and
        (140.1173,1009.5000) .. (214.2812,1009.5000) .. controls (288.4452,1009.5000)
        and (348.5625,949.3827) .. (348.5625,875.2188) .. controls (348.5625,801.0548)
        and (288.4452,740.9375) .. (214.2812,740.9375) -- cycle(244.5000,790.8438) ..
        controls (268.0981,791.3420) and (273.2571,821.4646) .. (288.2500,835.3125) ..
        controls (301.2515,856.3298) and (325.2448,871.8489) .. (328.5000,897.8438) ..
        controls (327.9493,921.0561) and (298.4384,924.0762) .. (290.2188,943.0000) ..
        controls (286.2271,963.2847) and (261.1112,968.6710) .. (251.7812,948.8750) ..
        controls (243.4201,932.7436) and (228.3908,913.1423) .. (241.7500,895.3750) ..
        controls (258.0833,877.6621) and (245.0726,855.1831) .. (231.8125,840.2812) ..
        controls (217.5359,827.1128) and (216.1269,796.0533) .. (239.1562,792.0000) --
        (241.7812,791.2500) -- (244.5000,790.8438) -- cycle;
      \path[draw=black,fill=ce1e1e1,line join=miter,line cap=butt,line width=0.800pt]
        (236.7177,908.3244) .. controls (229.4232,908.6311) and (221.9290,908.7907) ..
        (214.2857,908.7907) .. controls (177.2037,908.7907) and (143.6323,905.0331) ..
        (119.3314,898.9579) .. controls (95.0304,892.8827) and (80.0000,884.4898) ..
        (80.0000,875.2193);
      \path[draw=black,fill=ce1e1e1,line join=miter,line cap=butt,line width=0.800pt]
        (348.5714,875.2193) .. controls (348.5714,881.8608) and (340.8571,888.0519) ..
        (327.5463,893.2630);
    \end{scope}
    \begin{scope}[draw=black,fill=cffffff]
      \path[draw=black,fill=cffffff,line join=miter,line cap=butt,even odd rule,line
        width=0.800pt] (214.2812,740.9375) .. controls (140.1173,740.9375) and
        (80.0000,801.0548) .. (80.0000,875.2188) .. controls (80.0000,949.3827) and
        (140.1173,1009.5000) .. (214.2812,1009.5000) .. controls (288.4452,1009.5000)
        and (348.5625,949.3827) .. (348.5625,875.2188) .. controls (348.5625,801.0548)
        and (288.4452,740.9375) .. (214.2812,740.9375) -- cycle(244.5000,790.8438) ..
        controls (268.0981,791.3420) and (273.2571,821.4646) .. (288.2500,835.3125) ..
        controls (301.2515,856.3298) and (325.2448,871.8489) .. (328.5000,897.8438) ..
        controls (327.9493,921.0561) and (298.4384,924.0762) .. (290.2188,943.0000) ..
        controls (286.2271,963.2847) and (261.1112,968.6710) .. (251.7812,948.8750) ..
        controls (243.4201,932.7436) and (228.3908,913.1423) .. (241.7500,895.3750) ..
        controls (258.0833,877.6621) and (245.0726,855.1831) .. (231.8125,840.2812) ..
        controls (217.5359,827.1128) and (216.1269,796.0533) .. (239.1562,792.0000) --
        (241.7812,791.2500) -- (244.5000,790.8438) -- cycle;
      \path[draw=black,fill=cffffff,line join=miter,line cap=butt,line width=0.800pt]
        (236.7177,908.3244) .. controls (229.4232,908.6311) and (221.9290,908.7907) ..
        (214.2857,908.7907) .. controls (177.2037,908.7907) and (143.6323,905.0331) ..
        (119.3314,898.9579) .. controls (95.0304,892.8827) and (80.0000,884.4898) ..
        (80.0000,875.2193);
      \path[draw=black,fill=cffffff,line join=miter,line cap=butt,line width=0.800pt]
        (348.5714,875.2193) .. controls (348.5714,881.8608) and (340.8571,888.0519) ..
        (327.5463,893.2630);
    \end{scope}
    \path[draw=black,fill=c000003,line width=0.800pt] (350.9658,821.2189) --
      (363.2357,810.4041) -- (362.8860,816.4007) -- (367.3044,820.4701) -- cycle;
    \begin{scope}[draw=black,fill=cffffff]
      \path[draw=black,line join=miter,line cap=butt,even odd rule,line width=0.744pt]
        (214.2812,740.9026) .. controls (150.1404,740.9026) and (98.1478,801.0355) ..
        (98.1478,875.2188) .. controls (98.1478,949.4020) and (150.1404,1009.5350) ..
        (214.2812,1009.5350);
      \path[draw=black,line join=miter,line cap=butt,even odd rule,line width=0.653pt]
        (214.2812,740.8457) .. controls (164.8936,740.8457) and (124.8599,801.0041) ..
        (124.8599,875.2188) .. controls (124.8599,949.4334) and (164.8936,1009.5919)
        .. (214.2812,1009.5919);
      \path[draw=black,line join=miter,line cap=butt,even odd rule,line width=0.534pt]
        (214.2812,740.7711) .. controls (181.3108,740.7711) and (154.5849,800.9629) ..
        (154.5849,875.2188) .. controls (154.5849,949.4746) and (181.3108,1009.6665)
        .. (214.2812,1009.6665);
    \end{scope}
    \path[draw=black,line join=miter,line cap=butt,line width=0.800pt]
      (363.1498,815.9865) .. controls (399.7689,802.9139) and (415.9561,825.4672) ..
      (426.2844,789.7225);
    \path[draw=black,fill=black] (424.76917,787.70221) node[above right] (text4088)
      {$\mathcal{S}^3$};
  \end{scope}

\end{tikzpicture}

				\caption{Schematische Darstellung eines $\mathbb{R}\times\mathcal{S}^3$ Modells. Die Hohlkugeln ($\mathcal{S}^3$) symbolisieren das Modelluniversum zu verschiedenen Zeitpunkten.}
			\end{figure}
			Der Kugelradius soll eine Funktion von $t$ alleine sein. Wir betrachten also ein System mit der Metrik
			\begin{align}
				{\rm d}s^2=-N(t)^2 {\rm d}t^2+a(t)^2{\rm d}\sigma_3^2.
			\end{align}
			${\rm d}\sigma_3^2$ bezeichnet die Metrik der Einheitssphäre. Als Blockmatrix in der Standardbasis angeschrieben hat diese Metrik
			die Form 
			\begin{align}
				g=\left(
					\begin{array}{cc} 
						N(t)^2 & \vec{0}^T\\ 
						\vec{0} & a(t)^2 \gamma
					\end{array}\right),
			\end{align}
			wobei $\gamma$ die Metrik in der Einheitssphäre bezeichnet. Die im ADM-Formalismus besprochene Metrik der Hyperebene ist insofern
			\begin{align}
				h_{ij}=a(t)^2\gamma_{ij},\quad h^{ij}=\frac{1}{a(t)^2}\gamma^{ij},\quad \sqrt{h}=a^3\sqrt{\gamma}.
			\end{align}
			Aus dem Normalenvektor $n=(1/N,\vec{0}^T)^T$ folgt $n_0=(n^0)^{-1}=N$ und somit ergibt sich unmittelbar die extrinsische Krümmung nach $\label{def:extrcurv}$
			aus den Christoffelsymbolen \label{def:christ}:
			\begin{align}
				K_{ij}=-\nabla_i n_j=-\partial_i n_j+\Gamma^\mu_{ji}n_\mu=N\Gamma^0_{ji}=-\frac{1}{2N}\partial_t h_{ij}=-\frac{\dot{a}a}{N}\gamma_{ij}.
			\end{align}
			Dabei wurde verwendet, dass $g^{00}=N^{-2}$, $g_{0i}=0$ und $g_{ij}=h_{ij}$. Weiters kann $K_{ij}K^{ij}$ und ${\rm Tr}(K)$ berechnet werden:
			\begin{align}
				K_{ij}K^{ij}=&h^{ik}h^{jl}K_{kl}K_{ij}=\frac{1}{a(t)^4}\gamma^{ik}\gamma^{jl}\frac{a(t)^2\dot{a}(t)^2}{N(t)^2}\gamma_{kl}\gamma_{ij}=3\frac{\dot{a}(t)^2}{a(t)^2 N(t)^2}\\
				{\rm Tr}(K)=&h^{ij}K_{ij}=-\frac{\dot{a(t)}a(t)}{ N(t)}\frac{1}{a(t)}\gamma^{ij}\gamma_{ij}=3\frac{\dot{a}(t)}{a(t) N(t)}.
			\end{align}
			Für den Krümmungsskalar der 3-Sphäre mit Radius $a(t)$ gilt außerdem $\inHS R=6 a(t)^{-2}$.
			Wir betrachten desweiteren ein skalares massives Feld $\varphi(t)$, gekoppelt an das Gravitationsfeld,
			welches homogen auf der Sphäre definiert ist. Das Potental ist dann im einfachsten Fall
			\begin{align}
				V(\varphi)=\frac{m^2 \varphi^2}{2}.
			\end{align}
			Insofern folgt für die Wirkung, die hier um die kosmologische Konstante erweitert wurde:
			\begin{align}
				S_{EH}=&\frac{1}{16\pi G}\int{\rm d}t\int_{\mathcal{S}^3}{\rm d}^3 x\, N\sqrt{h}\left(\inHS R-2\Lambda+{\rm Tr}(K)^2-K^{ij}K_{ij}+\right.\notag\\
						&\qquad\qquad\qquad\quad\qquad\qquad\left.+\frac{1}{2}g^{\mu\nu}\partial_\mu \varphi\partial_\nu \varphi-V(\varphi)\right)=\notag\\
					&\frac{1}{16\pi G}\int{\rm d}t\, N a^3\left(\frac{6}{a^2}-2\Lambda-6\frac{\dot{a}^2}{a^2N^2}+\frac{1}{2 N^2}\dot{\varphi}^2-V(\varphi)\right) \VSthree\\
					\Rightarrow S_{EH} \propto& \int{\rm d}t\, N a^3\left(\frac{6}{a^2}-2\Lambda-6\frac{\dot{a}^2}{a^2N^2}+\frac{1}{2 N^2}\dot{\varphi}^2-V(\varphi)\right).
			\end{align}
			Die Zeitableitungen wurden der Übersichtlichkeit halber weggelassen. Da außerdem die einzige Größe, die von $\vec{x}$ abbhängt $\sqrt{\gamma}$ ist
			und diese desweiteren unabhängig von $N,\ a$ und $\dot{a}$ ist, verhällt sie sich in weiterer betrachtung als Konstante (Oberfläche der Einheitssphäre) und
			wird mit $\VSthree$ notiert. $\Lambda$ ist die kosmologische Konstante, die in den vorigen Kapiteln immer gleich null gestzt wurde.
			
			Der nächste Schritt ist es, sich die Hamiltonfunktion zu berechnen. Für die konjugierten Impuls ergibt sich sofort
			\begin{align}
				p_a=\frac{\partial L}{\partial \dot{a}}=-12\frac{a\dot{a}}{N},\quad p_\varphi=\frac{\partial L}{\partial \dot{\varphi}}=\frac{a^3\dot{\varphi}}{N}.
			\end{align}
			Daher ist die Hamiltonfunktion von $L$:
			\begin{align*}
				H=&p_a \dot{a}+p_\varphi \dot{\varphi}-L=N H_\perp\propto\\
				\propto&N\left(\frac{p_a^2}{24a}-\frac{p_\varphi}{2a^3}-6a+2a^3\Lambda+V(\varphi)a^3\right).
			\end{align*}
			Aus den Betrachtungen im Kapitel über den ADM Formalismus erwarten wir, dass
			\begin{align*}
				H_\perp=\frac{p_a^2}{24a}-\frac{p_\varphi^2}{2a^3}-6a+2a^3\Lambda+V(\varphi)a^3
			\end{align*}
			verschwindet. Tatsächlich lässt sich dies auch leicht mithilfe der Euler-Langrangegleichung zeigen. Bildet man nämlich
			\begin{align*}
				\frac{{\rm d}}{{\rm d}t}\frac{\partial L}{\partial \dot{N}}-\frac{\partial L}{\partial N}=0
			\end{align*}
			ergibt sich, wobei die Ableitung $\frac{\partial L}{\partial \dot{N}}$ offensichtlich verschwindet:
			\begin{align*}
				0=\frac{{\rm d}L}{{\rm d } N}&=6a-2a^3 \Lambda+6\frac{a\dot{a}^2}{N^2}-\frac{a^3\dot{\varphi}^2}{2N^2}-V(\varphi)a^3=\\
				&=\frac{p_a^2}{24a}-\frac{p_\varphi^2}{2a^3}-6a+2a^3\Lambda+V(\varphi)a^3=H_\perp.
			\end{align*}
			Desweiteren lässt sich an diesem Modell auch Satz \ref{thm:full} verdeutlichen. Bildet man die verbleibenden Bewegungsgleichungen
			mit der Eichung $N=1$ erhällt man unmittelbar:
			\begin{align}
				\dot{a}^2+12+a^2(-\dot{\varphi}^2+2V(\varphi)+4\Lambda)&=0\\
				\ddot{\varphi}+3\frac{\dot{a}}{a}\dot{\varphi}+V^\prime(\varphi)&=0\\
				\ddot{a}+\frac{1}{2}\left(\frac{\dot{a}}{a}-\frac{1}{a}+a\Lambda-\frac{a\dot{\varphi}^2}{4}+\frac{V(\varphi)a}{2}\right)&=0
			\end{align}
			Differenziert man die erste Gleichung und setz zweite ein, ist das Resultat die dritte Gleichung.
			
			Der super-Hamilton constraint für das gewählte Minisuperspace Modell ist somit
			\begin{align}
				H_\perp=p_a^2-12p_\varphi^2-a^2(144+48 a^2 \Lambda+24V(\varphi)a^3)=0,
			\end{align}
			was nach Reskalierung von $N$, $\Lambda$ und $V(\varphi)$ equivalent zu der Form
			\begin{align}
				H_\perp=p_a^2-p_\varphi^2-a^2(1+a^2 (\Lambda+V(\varphi)))=0
			\end{align}
			ist. Die Wheeler-DeWitt Gleichung für ein solches System ist -- nach Ersetzung von $p_a=-i\veps\frac{\partial}{\partial a}$ und $p_\varphi=-i\veps\frac{\partial}{\partial \varphi}$; die Bedeutung von $\veps$ wird noch klar werden -- eine
			partielle Differentialgleichung zweiter Ordnung:
			\begin{align}
				\left(-\veps^2\frac{\partial^2}{\partial a^2}+\veps^2\frac{\partial^2}{\partial \varphi^2}-a^2(1+a^2 (\Lambda+V(\varphi))\right)\Psi(a,\varphi)=0,
			\end{align}
			beziehungsweise ohne skalares Feld eine Gleichung, ähnlich der stationären Schrödingergleichung:
			\begin{align}
				\left(-\veps^2\frac{{\rm d}^2}{{\rm d}a^2}-a^2(1+a^2\Lambda)\right)\Psi(a)=0.
			\end{align}
			Da $\Psi$ die Wellenfunktion diese einfachen Universums darstellt, müsste sich in dieser Gleichung 
			irgendwo ein Zeitbegriff verstecken. Analysieren kann man eine solche Gleichung beispielsweise mithilfe einer
			analytischen Näherung wie der WKB Methode. Das immer stärker werdende Oszillieren der Amplitude wäre
			beispielsweise eine mögliche Manifestation der Zeit in diesem Modell.
		\subsection{Zeitfreie Theorien}
			Der Vollständigkeit halber ist es an dieser Stelle noch notwendig zu Erwähnen,
			dass ein vollständig zeitfreies Modell ebenfalls ein valider Kandidat für eine Theorie der Quantengravitation darstellt. In ihm ist das Universum
			nicht an eine Zeit gebunden, sie präsentiert sich für unsere Auffassung nur als phänomenologisches 
			Konzept. Das Problem gestalltet sich hier allerdings auch nit weniger schwierig,
			da das Konstruieren eines Zeitoperators und das Angeben eines geeigeneten Hilberraumes nicht
			ohne weiters durchgeführt werden kann \cite{qg06}.
	\bibliography{handout.bib}{}
	\bibliographystyle{plain}
\end{document}
