\documentclass[mathserif]{beamer}
\usepackage[utf8]{inputenc}
\usepackage[T1]{fontenc}
\usepackage[ngerman,german]{babel}
\usepackage{amsmath}
\usepackage{amsthm}
\usepackage{amssymb}
\usepackage{tikz}
\usepackage{bbold}
\usepackage{enumerate}
\usepackage{ifthen}
\usepackage{breqn}
\usepackage{floatrow}
\usetikzlibrary{shapes,shadows,arrows,decorations.markings}
\usetikzlibrary{decorations.pathreplacing,calc}
\usetheme{Szeged}
\usecolortheme{seagull}

\newcommand{\inftyint}{\int_{-\infty}^{\infty}}
\newcommand{\intR}{\int_{\mathbb{R}}}
\newcommand{\intRRR}{\int_{\mathbb{R}^3}}
\newcommand{\opd}[2][\vphantom{1}]{\!\operatorname{d}^{#1}\! #2\ }
\newcommand{\opdpath}{\opd{\mu(p)}}
\newcommand{\real}{\operatorname{Re}}
\newcommand{\imag}{\operatorname{Im}}
\newcommand{\residue}{\operatorname{Res}}
\newcommand{\veps}{\varepsilon}
\newcommand{\vx}{\vec{x}}
\newcommand{\vp}{\vec{p}}
\newcommand{\vacuumt}{\langle 0 |}
\newcommand{\vacuum}{| 0 \rangle}
\newcommand{\Res}{\operatorname{Res}}
\newcommand{\inHS}{{}^{(3)}\!}
\newcommand{\VSthree}{\mu(\mathcal{S}^3)}

\title{\textbf{Das Problem der Zeit in der Quantengravitation}}
\author{\small {Teil 2: Quantengravitation und die Identifikation der Zeit -- Minisuperspace Modelle und Ausblick auf weitere Interpretationen}}
\institute{}
\date{\today}

\theoremstyle{definition}

\begin{document}
	\begin{frame}
		\titlepage
	\end{frame}

	\begin{frame}{Überblick}
		\tableofcontents
	\end{frame}

\section{Kanonische Quantisierung}
\subsection{Wiederholung am skalaren Feld}
	\begin{frame}{Wiederholung am skalaren Feld}
		Lagrangedichte für massives skalares Feld:
		\begin{align}
			\mathcal{L}=\frac{1}{2}(\partial_\mu\varphi \partial^\mu\varphi-m^2\varphi^2)
		\end{align}
		\pause
		\textbf{Kanonische Quantisierung:} Feldgrößen werden zu Operatoren\\
		\pause
		\textbf{Kommutatorrelationen:}
		\begin{align}
			&[\hat{\varphi}(\vec{x}), \hat{\varphi}(\vec{y})]=0=[\hat{\pi}(\vec{x}), \hat{\pi}(\vec{y})] \\
			&[\hat{\varphi}(\vec{x}), \hat{\pi}(\vec{y})]=i\delta(\vec{x}-\vec{y}).
		\end{align}
		Ableitungen werden zu Funktionalableitungen (z.B. Impulsoperator: wirkt auf Wellenfunktion im Ortsbild wie Ableitungsoperator)
	\end{frame}
\subsection{Quantisierung der Constraints:}
	\begin{frame}{Quantisierung der Constraints:}
		$h_ij$ und $\pi_{ij}$ in den Rang von Operatoren erhoben.
		\pause
		Die Constraints 
		\begin{align}
			&H_{\perp}:=16\pi G G_{ijkl}\pi^{ij}\pi^{kl}-\frac{1}{16\pi G}\sqrt{h}\inHS R=0 \label{equ:constraint1}\\
			&H^i:=-2\inHS\nabla_j\pi^{ij}=0 \label{equ:constraint2}
		\end{align}
		\pause
		\begin{center}
			Wenden Quantisierungsregeln an
		\end{center}
		\pause
		Für generalisierte Kommutator von Impulse \& Metrik:
		\begin{align}
			[h_{ij}(\vec{x}),\pi^{kl}(\vec{x}^{\,\prime})]=\delta^k_{(i}\delta^l_{j)}\delta(\vec{x}-\vec{x}^{\,\prime})
		\end{align}
		\pause
		Die Constrains wirken auf Wellenfunktionen:
		\begin{align}
			\hat{H}_\perp\Psi[h]&=0\\
			\hat{H}_i\Psi[h]&=0.
		\end{align}
	\end{frame}
	\begin{frame}
		Für super-Hamilton constraint:
		\begin{align}
			\boxed{-16\pi G G_{ijkl}\frac{\delta^2\Psi[h]}{\delta h_{ij}\delta h_{kl}}-\frac{1}{16\pi G}\inHS R\Psi[h]=0}
		\end{align}
		\begin{center}
			\textbf{Wheeler-DeWitt Gleichung}
		\end{center}
		\pause
		Zentrales Objekt in der kanonischen Quantisierung der Raumzeit!
	\end{frame}
\section{Problem der Zeit}
	\begin{frame}
		In Wheeler-DeWitt Gleichung: keine explizite Zeit!
		In allen Feldtheorien gibt es jedoch bisher einen universellen Zeitparameter
		\pause
		\begin{center}Wirft neue Fragen auf:\end{center}
		\begin{itemize}
			\item Kann die Zeit zurückgewonnen werden?
			\item Wenn nein, ist eine Quantentheorie ohne Zeit mit diesen Mitteln sinnvoll modellierbar?
		\end{itemize}
	\end{frame}
	\begin{frame}
		Verschiedene Ansätze möglich:
		\begin{itemize}
			\item Die Zeit muss vor der Quantisierung spezifiziert werden.
			\pause
			\item Die Zeit muss nach der Quantisierung spezifiziert werden.
			\pause
			\item Die Zeit ist keine allumfassendes Konzept in einer Theorie der Quantengravitation
				und tritt nur phänomenologisch auch
			\pause
		\end{itemize}
		Alle Ansätze haben ihre eigenen Vor- und Nachteile.
	\end{frame}
	%% TODO: finish %%
\end{document}
